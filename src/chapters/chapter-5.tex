\chapter{Simpulan dan Saran}

% TODO: tambah kasus ICPC, yang ditanyain turfa waktu seminar 2

\par Terdapat beberapa kesimpulan dan saran terkait pengembangan dan hasil pada tugas akhir ini. Pada bab ini diberikan kesimpulan dari tugas akhir dan saran pengembangan lebih lanjut terhadap solusi yang sudah dipaparkan.

\section{Simpulan}

\par Pada tugas akhir ini telah dikembangkan sistem \textit{online judge} yang bernama UGrade. Beberapa kesimpulan yang didapatkan pada pengerjaan tugas akhir ini antara lain adalah:
\begin{enumerate}
    \item Terdapat peningkatan kinerja penilaian yang cukup tinggi pada UGrade dibandingkan dengan sistem \textit{online judge} yang saat ini populer digunakan dalam menyelenggarakan \textit{competitive programming}.
    \item Kinerja penilaian sistem \textit{online judge} UGrade tidak dipengaruhi oleh jumlah peserta, akan tetapi dipengaruhi oleh nilai \textit{grading size} dari kompetisi.
    \item Nilai \textit{grading size} yang tinggi mengakibatkan berkurangnya kinerja penilaian sistem \textit{online judge} UGrade akan tetapi meningkatkan keamanannya. Begitu juga sebaliknya, nilai \textit{grading size} yang rendah meningkatkan kinerja penilaian tetapi mengurangi tingkat keamanannya.
    \item Berdasarkan hasil pengujian keamanan, kemanan dari sistem \textit{online judge} UGrade cukup untuk menyelenggarakan kompetisi \textit{competitive programming}.
    \item UGrade dapat menjaga keadilan penilaian meskipun komputer \textit{worker} memiliki spesifikasi yang berbeda-beda. Hal ini dicapai dengan cara membandingkan hasil eksekusi solusi juri dan eksekusi solusi peserta pada komputer \textit{worker}.
    \item UGrade masih belum bisa menangani serangan yang berbentuk \textit{reverse engineering}. Meskipun begitu, beberapa kompetisi \textit{competitive programming} dapat menangani masalah ini jika peserta menggunakan komputer yang disediakan oleh juri, misalnya seperti pada babak final dari kompetisi ACM-ICPC.
    \item Teknik \textit{load balancing} yang digunakan oleh UGrade menyebabkan beban UGServer menjadi sangat berat karena \textit{worker} selalu mengirimkan \textit{request} kepada \textit{server} secara periodik.
    \item Terdapat kasus yang masih belum bisa ditangani secara akurat oleh sistem \textit{sandbox} UGrade yaitu pembatasan jumlah proses yang berjalan dalam lingkungan \textit{sandbox} dan pembatasan hak akses terhadap \textit{file} yang ada dalam lingkungan \textit{sandbox}.
    \item Penggunaan istilah \textit{grading group}, \textit{grading}, \textit{grading size} dan \textit{grader group} sangat membingungkan dan sebaiknya istilah tersebut diganti dengan menggunakan istilah yang lebih deskriptif.
\end{enumerate}

% \par Pada tugas akhir ini telah diciptakan sistem \textit{online judge} yang bernama \textit{UGrade}. \textit{UGrade} memanfaatkan komputer pengguna untuk menjadi \textit{worker} dalam melakukan penilaian jawaban peserta kompetisi. Berdasarkan hasil pengujian, penggunaan komputer pengguna sebagai \textit{worker} meningkatkan kinerja penilaian. Hal ini dikarenakan jumlah \textit{worker} yang melakukan penilaian sebanding dengan jumlah peserta yang mengirimkan \textit{jawaban}.

% \par Pada tugas akhir ini telah diciptakan sistem \textit{online judge} yang bernama UGrade. UGrade memanfaatkan komputer pengguna untuk menjadi \textit{worker} dalam melakukan penilaian jawaban peserta kompetisi. Berdasarkan hasil pengujian, UGrade memiliki kinerja penilaian jawaban peserta yang lebih tinggi dibandingkan sistem \textit{online judge} yang saat ini populer digunakan. Pada sistem \textit{online judge} UGrade, kinerja dari proses penilaian jawaban peserta dipengaruhi oleh nilai \textit{grading size} yang digunakan dalam kompetisi. Nilai \textit{grading size} yang tinggi akan mengurangi kinerja dari proses penilaian, akan tetapi dapat meningkatkan keamanan sistem. Begitu juga sebaliknya, nilai \textit{grading size} yang rendah meningkatkan kinerja penilaian, akan tetapi dapat mengurangi keamanan sistem. Juri perlu mengatur nilai \textit{grading size} sehingga proses penilaian cukup aman dan memiliki kinerja yang tinggi.

% \par Dalam kompetisi \textit{competitive programming}, peserta mungkin saja melakukan serangan yang mengakibatkan kerusakan pada sistem \textit{online judge}. Keamanan dari sistem \textit{online judge} yang dibangun perlu diuji untuk menentukan ketahanannya terhadap serangan dari peserta. Pengujian terhadap serangan yang mungkin dilakukan oleh peserta dilakukan dengan mengirimkan jawaban yang bebahaya kepada sistem yang dibangun. Keamanan dari UGrade ditentukan berdasarkan adanya efek samping yang muncul ketika jawaban peserta dinilai. Berdasarkan pengujian yang telah dilakukan, tidak ada efek samping yang muncul setelah jawaban peserta yang berbahaya dinilai. Oleh karena itu, sistem UGrade dapat dikatakan cukup aman untuk digunakan dalam kompetisi \textit{competitive programming}.

% \par Kebenaran dari sistem \textit{online judge} UGrade diuji dengan mengirimkan beberapa jenis jawaban secara berulang-ulang kepada sistem. Kebenaran dari sistem UGrade ditentukan berdasarkan \textit{verdict} yang diberikan oleh sistem terhadap jawaban-jawaban tersebut. Pengujian ini dilakukan dengan menggunakan tujuh jenis jawaban yang berbeda. Berdasarkan hasil pengujian, UGrade memberikan \textit{verdict} yang benar kepada seluruh jawaban yang dikirimkan oleh peserta.

% \par Sistem \textit{online judge} UGrade masih belum dapat menangani serangan yang berupa \textit{reverse engineering}. Peserta yang memiliki akses \textit{root} dari komputer yang digunakannya dapat melakukan serangan terhadap sistem dengan mengubah kode program \textit{autograder} dari UGrade. Meskipun begitu, untuk beberapa jenis kompetisi \textit{competitive programming}, serangan ini dapat diatasi dengan tidak memberikan akses \textit{root} kepada peserta. Kompetisi ACM-ICPC merupakan salah satu jenis kompetisi yang dapat bertahan dari serangan \textit{reverse engineering}.

% \par Pada kompetisi ACM-ICPC, peserta umumnya akan diseleksi terlebih dahulu secara bertingkat sebelum akhirnya berkompetisi di \textit{world final}. Sebelum babak \textit{world final}, jumlah peserta yang mengikuti kompetisi relatif sedikit sehingga sistem \textit{online judge} yang saat ini sering digunakan cukup untuk menjalankan kompetisi tersebut. Pada \textit{world final}, jumlah peserta kompetisi menjadi sangat banyak karena terdiri dari peserta-peserta yang lolos dari berbagai negara. Pada \textit{world final}, juri akan memberikan komputer khusus kepada peserta untuk mengikuti kompetisi tersebuti. Pada babak \textit{world final}, juri dapat menggunakan sistem \textit{online judge} UGrade untuk menilai jawaban peserta. Juri dapat mengatur agar peserta tidak mendapatkan akses \textit{root} pada komputer yang digunakannya sehingga tidak dapat melakukan berbagai jenis serangan yang berupa \textit{reverse engineering}.

% \par Teknik \textit{load balancing} yang digunakan oleh UGrade mengikuti teknik \textit{load balancing} yang digunakan oleh sistem \textit{online judge} yang saat ini populer digunakan yaitu \textit{pull-based load balancing}. Teknik \textit{load balancing} ini ternyata tidak terlalu baik digunakan jika jumlah \textit{worker} sangat banyak. \textit{Worker} yang banyak mengakibatkan banyaknya \textit{request} yang harus ditangani oleh \textit{server} sehingga \textit{server} menjadi mudah gagal. Hal ini tidak ditemukan pada sistem \textit{online judge} yang saat ini populer digunakan. Hal tersebut dikarenakan umumnya sistem \textit{online judge} yang saat ini populer digunakan tidak menggunakan banyak \textit{worker} untuk melakukan penilaian.

% \par Sistem \textit{sandbox} yang digunakan pada UGrade dibangun dengan bahasa Go dan diberi nama UGSbox. Saat ini, UGSbox dapat memberikan isolasi yang cukup baik untuk mengeksekusi program yang dikirimkan oleh peserta. Meskipun begitu, masih terdapat beberapa kekurangan pada UGSbox karena dibangun dengan bahasa Go. Salah satu kekurangan UGSbox adalah tidak bisa membatasi jumlah proses yang diciptakan oleh proses yang diisolasi secara akurat. Hal ini disebabkan bahasa Go memiliki \textit{runtime} yang perlu dijalankan sebagai proses tersendiri sehingga UGSbox sulit untuk membatasi jumlah proses yang harus diisolasi. Selain itu, \textit{resource} yang digunakan oleh \textit{go runtime} juga ikut dihitung dalam membatasi \textit{resource} proses yang diisolasi. Hal tersebut mengakibatkan pembatasan \textit{resource} yang digunakan oleh proses yang diisolasi menjadi kurang akurat.

\section{Saran}

\par Terdapat beberapa kekurangan pada pengerjaan tugas akhir ini. Berikut beberapa saran yang dapat diberikan untuk pengerjaan tugas akhir ini:
\begin{enumerate}
    \item Pemilihan teknik penilaian membutuhkan adanya studi lebih lanjut. Jika banyak peserta melakukan serangan terhadap sistem, \textit{self grading} dapat mengurangi risiko terjadinya kerusakan sistem. Sistem \textit{online judge} dapat dibuat untuk mudah dikonfigurasi (\textit{configurable}) sehingga dapat menggunakan beberapa jenis sistem penilaian.
    \item UGDesktop berukuran sangat besar karena mengguna \textit{framework} React dan Electron. Aplikasi \textit{desktop} dapat dibuat lebih \textit{native} dengan menggunakan \textit{library} seperti GTK.
    \item Pemilihan \textit{framework} Django untuk bertindak sebagai \textit{server} tidak dapat menangani banyaknya pengguna dengan efisien. Bahasa Go dapat digunakan untuk meningkatkan kinerja dari \textit{server}.
    \item Karena setiap peserta berperan sebagai \textit{worker}, jumlah \textit{request} yang perlu ditangani oleh UGServer sangat banyak dan memberatkan UGServer. Hal tersebut dikarenakan metode yang digunakan untuk \textit{load balancing} adalah \textit{pull-based load balancing} sehingga setiap \textit{worker} terus menerus mengirim \textit{request} kepada \textit{server}. Metode \textit{load balancing} perlu dimodifikasi sehingga beban \textit{server} dapat berkurang.
    \item Penggunaan bahasa Go untuk mengembangkan lingkungan \textit{sandbox} kurang efektif. Bahasa Go sulit untuk melakukan \textit{system call fork} dan \textit{exec} secara terpisah karena adanya \textit{go runtime} yang perlu dijalankan. \textit{Sandbox} dapat dikembangkan dengan bahasa yang lebih \textit{low-level} seperti C, C++ atau Rust. 
    \item Penggunaan istilah \textit{grading group}, \textit{grading size}, \textit{grading} dan \textit{grader group} sulit dipahami. Penggunaan nama yang mirip sebaiknya dihindari dengan memilih nama yang lebih deskriptif dan tidak terlalu mirip. 
    \item Bahasa yang didukung oleh UGrade masih terbatas C dan C++. UGrade perlu memberikan dukungan terhadap bahasa Pascal, Python dan Java karena bahasa tersebut sering digunakan pada kompetisi \textit{competitive programming}.
\end{enumerate}
