\chapter{Simpulan dan Saran}

% TODO: tambah kasus ICPC, yang ditanyain turfa waktu seminar 2

\par Pada bab ini diberikan kesimpulan dari Tugas Akhir dan saran pengembangan kebih lanjut terhadap solusi yang sudah dipaparkan.

\section{Simpulan}

% \par Pada tugas akhir ini telah diciptakan sistem \textit{online judge} yang bernama \textit{UGrade}. \textit{UGrade} memanfaatkan komputer pengguna untuk menjadi \textit{worker} dalam melakukan penilaian jawaban peserta kompetisi. Berdasarkan hasil pengujian, penggunaan komputer pengguna sebagai \textit{worker} meningkatkan kinerja penilaian. Hal ini dikarenakan jumlah \textit{worker} yang melakukan penilaian sebanding dengan jumlah peserta yang mengirimkan \textit{jawaban}.

\par Pada tugas akhir ini telah diciptakan sistem \textit{online judge} yang bernama \textit{UGrade}. \textit{UGrade} memanfaatkan komputer pengguna untuk menjadi \textit{worker} dalam melakukan penilaian jawaban peserta kompetisi. Peningkatan kinerja dari sistem yang dibangun belum dapat ditentukan karena pengujian belum dilakukan.

\par Pengujian terhadap serangan yang mungkin dilakukan oleh peserta dilakukan dengan mengirimkan jawaban yang bebahaya kepada sistem yang dibangun. Keamanan dari sistem belum dapat ditentukan karena pengujian belum ditentukan.

\par Kebenaran dari sistem penilaian diuji dengan mengirimkan jawaban secara berulang-ulang dengan harapan \textit{autograder} memerikan \textit{verdict} yang sama. Kebenaran dari sistem yang dibangun belum dapat ditentukan karena pengujian belum dilakukan.

\section{Saran}

\par Saran yang dapat diberikan untuk pengembangan terkait pengerjaan tugas akhir ini adalah sebagai berikut:
\begin{enumerate}
    \item Pemilihan teknik penilaian membutuhkan adanya studi lebih lanjut. Jika banyak peserta melakukan serangan terhadap sistem, \textit{self grading} dapat mengurangi risiko terjadinya kerusakan sistem. Sistem \textit{online judge} dapat dibuat mudah untuk dikonfigurasi sehingga dapat menggunakan beberapa jenis sistem penilaian.
    \item \textit{UGDesktop} berukuran sangat besar karena mengguna \textit{framework} React dan Electron. Aplikasi \textit{desktop} dapat dibuat lebih \textit{native} dengan menggunakan \textit{library} seperti GTK.
    \item Pemilihan \textit{framework} Django untuk bertindak sebagai \textit{server} tidak dapat menangani banyaknya pengguna dengan efisien. Bahasa Go dapat digunakan untuk meningkatkan kinerja dari \textit{server}.
    \item Penggunaan bahasa Go untuk mengembangkan lingkungan \textit{sandbox} kurang efektif. Bahasa Go sulit untuk melakukan \textit{system call fork} dan \textit{exec} secara terpisah karena adanya \textit{go runtime} yang perlu dijalankan. \textit{Sandbox} dapat dikembangkan dengan bahasa yang lebih \textit{low-level} seperti C, C++ atau Rust. 
    \item Penggunaan istilah \textit{grading group}, \textit{grading size}, \textit{grading} dan \textit{grader group} sulit dipahami. Penggunaan nama yang mirip sebaiknya dihindari dengan memilih nama yang lebih deskriptif dan tidak terlalu mirip. 
    \item Bahasa yang didukung oleh \textit{UGrade} masih terbatas C dan C++. \textit{UGrade} perlu memberikan dukungan terhadap bahasa Pascal, Python dan Java karena bahasa tersebut sering digunakan pada kompetisi \textit{competitive programming}.
\end{enumerate}